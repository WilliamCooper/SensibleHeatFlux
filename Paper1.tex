\batchmode
\makeatletter
\def\input@path{{/run/media/cooperw/SeagateDesktop/WACspace/RStudio/SensibleHeatFlux/}}
\makeatother
\documentclass[11pt,twoside,american,12pt,twoside,american]{article}\usepackage[]{graphicx}\usepackage[]{color}
% maxwidth is the original width if it is less than linewidth
% otherwise use linewidth (to make sure the graphics do not exceed the margin)
\makeatletter
\def\maxwidth{ %
  \ifdim\Gin@nat@width>\linewidth
    \linewidth
  \else
    \Gin@nat@width
  \fi
}
\makeatother

\definecolor{fgcolor}{rgb}{0.345, 0.345, 0.345}
\newcommand{\hlnum}[1]{\textcolor[rgb]{0.686,0.059,0.569}{#1}}%
\newcommand{\hlstr}[1]{\textcolor[rgb]{0.192,0.494,0.8}{#1}}%
\newcommand{\hlcom}[1]{\textcolor[rgb]{0.678,0.584,0.686}{\textit{#1}}}%
\newcommand{\hlopt}[1]{\textcolor[rgb]{0,0,0}{#1}}%
\newcommand{\hlstd}[1]{\textcolor[rgb]{0.345,0.345,0.345}{#1}}%
\newcommand{\hlkwa}[1]{\textcolor[rgb]{0.161,0.373,0.58}{\textbf{#1}}}%
\newcommand{\hlkwb}[1]{\textcolor[rgb]{0.69,0.353,0.396}{#1}}%
\newcommand{\hlkwc}[1]{\textcolor[rgb]{0.333,0.667,0.333}{#1}}%
\newcommand{\hlkwd}[1]{\textcolor[rgb]{0.737,0.353,0.396}{\textbf{#1}}}%
\let\hlipl\hlkwb

\usepackage{framed}
\makeatletter
\newenvironment{kframe}{%
 \def\at@end@of@kframe{}%
 \ifinner\ifhmode%
  \def\at@end@of@kframe{\end{minipage}}%
  \begin{minipage}{\columnwidth}%
 \fi\fi%
 \def\FrameCommand##1{\hskip\@totalleftmargin \hskip-\fboxsep
 \colorbox{shadecolor}{##1}\hskip-\fboxsep
     % There is no \\@totalrightmargin, so:
     \hskip-\linewidth \hskip-\@totalleftmargin \hskip\columnwidth}%
 \MakeFramed {\advance\hsize-\width
   \@totalleftmargin\z@ \linewidth\hsize
   \@setminipage}}%
 {\par\unskip\endMakeFramed%
 \at@end@of@kframe}
\makeatother

\definecolor{shadecolor}{rgb}{.97, .97, .97}
\definecolor{messagecolor}{rgb}{0, 0, 0}
\definecolor{warningcolor}{rgb}{1, 0, 1}
\definecolor{errorcolor}{rgb}{1, 0, 0}
\newenvironment{knitrout}{}{} % an empty environment to be redefined in TeX

\usepackage{alltt}
\usepackage{mathptmx}
\usepackage{helvet}
\renewcommand{\ttdefault}{lmtt}
\renewcommand{\familydefault}{\rmdefault}
\usepackage[T1]{fontenc}
\usepackage[letterpaper]{geometry}
\geometry{verbose,tmargin=1in,bmargin=1in,lmargin=1.2in,rmargin=1in}
\usepackage{fancyhdr}
\pagestyle{fancy}
\setcounter{tocdepth}{2}
\setlength{\parskip}{\medskipamount}
\setlength{\parindent}{0pt}
\usepackage{color}
\definecolor{page_backgroundcolor}{rgb}{1, 1, 1}
\pagecolor{page_backgroundcolor}
\usepackage{babel}
\usepackage{float}
\usepackage{url}
\usepackage{amsmath}
\usepackage{esint}
\usepackage[authoryear]{natbib}
\PassOptionsToPackage{normalem}{ulem}
\usepackage{ulem}
\usepackage[unicode=true,
 bookmarks=true,bookmarksnumbered=false,bookmarksopen=false,
 breaklinks=true,pdfborder={0 0 0},pdfborderstyle={},backref=section,colorlinks=true]
 {hyperref}
\hypersetup{pdftitle={The Delayed Response of Airborne Thermometers, Part 1},
 pdfauthor={William A. Cooper},
 pdfsubject={Assessment of Measurements of Sensible-Heat Flux},
 pdfkeywords={sensible heat flux, temperature, time response of sensors, NCAR Research Aviation Facility, research aircraft, NCAR/EOL/RAF},
  linkcolor=blue, citecolor={blue}}

\makeatletter

%%%%%%%%%%%%%%%%%%%%%%%%%%%%%% LyX specific LaTeX commands.
%% Because html converters don't know tabularnewline
\providecommand{\tabularnewline}{\\}

%%%%%%%%%%%%%%%%%%%%%%%%%%%%%% User specified LaTeX commands.
\definecolor{fgcolor}{rgb}{0.345, 0.345, 0.345}\definecolor{messagecolor}{rgb}{0, 0, 0}\definecolor{warningcolor}{rgb}{1, 0, 1}\definecolor{errorcolor}{rgb}{1, 0, 0}\setlength{\headheight}{14.5pt}\usepackage{babel}


\definecolor{fgcolor}{rgb}{0.345, 0.345, 0.345}\definecolor{messagecolor}{rgb}{0, 0, 0}\definecolor{warningcolor}{rgb}{1, 0, 1}\definecolor{errorcolor}{rgb}{1, 0, 0}\usepackage{babel}
% macro for italic page numbers in the index
\newcommand{\IndexDef}[1]{\textit{#1}}
\newcommand{\IndexPrimary}[1]{\textbf{#1}}
% force a page break at the start of sections
\let\stdsection\section
\renewcommand{\section}{\newpage\stdsection}


% workaround for a makeindex bug,
% see sec. "Index Entry Order"
% only uncomment this when you are using makindex
%\let\OrgIndex\index 
%\renewcommand*{\index}[1]{\OrgIndex{#1}}
%\usepackage{splitidx}

% workaround for a makeindex bug,
% see sec. "Index Entry Order"
% only uncomment this when you are using makindex
\let\OrgIndex\index 
\renewcommand*{\index}[1]{\OrgIndex{#1}}
\usepackage{splitidx}
%\indexsetup{noclearpage}
\AtBeginDocument{
  \def\labelitemii{\(\circ\)}
  \def\labelitemiii{\(\triangleright\)}
}
\usepackage[font={normal,sl}]{caption}% set captions slanted

\IfFileExists{upquote.sty}{\usepackage{upquote}}{}

\newenvironment{lylist}[1]{\begin{list}{}
{\settowidth{\labelwidth}{#1}
\setlength{\leftmargin}{\labelwidth}
\addtolength{\leftmargin}{\labelsep}
\renewcommand{\makelabel}[1]{##1\hfil}}}{\end{list}}
\newcommand{\datetoday}{\number\day\space
     \ifcase\month\or January\or February\or March\or April\or May\or
     June\or July\or August\or September\or October\or November\or
     December\fi
     \space\number\year}
\newcommand{\EOLmemo}{\null \vskip-1.5truein
{\raggedright \textsf{\textsc{\large \textcolor{blue}{Earth Observing Laboratory}}}}\par
{\raggedright \textsf{\textsl{\textcolor{blue}{Memorandum:}}}} \par \vskip6pt
{\color{blue}{\hrule}}\par
\vskip0.3truein \leftline{\hskip \longindent \datetoday} \vskip0.2truein
\thispagestyle{empty}}
\newcommand{\attachm}[1]{\begin{lylist}{Attachments:00}
\item [Attachments:] {#1}
\end{lylist}}
\newcommand{\cc}[1]{\begin{lylist}{Attachments:00}
\item [cc:] {#1}
\end{lylist}}
\newcommand{\attach}[1]{\begin{lylist}{Attachments:00}
\item [Attachment:] {#1}
\end{lylist}}

\IfFileExists{upquote.sty}{\usepackage{upquote}}{}
\pagenumbering{gobble}
%\usepackage{xcolor}
%\providecolor{lyxadded}{rgb}{0,0,1}
%\providecolor{lyxdeleted}{rgb}{1,0,0}
%% Change tracking with ulem
%\DeclareRobustCommand{\lyxadded}[3]{{\color{lyxadded}{}#3}}
%\DeclareRobustCommand{\lyxdeleted}[3]{{\color{lyxdeleted}\sout{#3}}}

\makeatother
\IfFileExists{upquote.sty}{\usepackage{upquote}}{}
\begin{document}
\title{The Delayed Response of Airborne Thermometers:\\
Part 1: Determining the Transfer Function}
\author{William A. Cooper and others...}
\date{\textcolor{red}{DRAFT} April 2020}

\maketitle
National Center for Atmospheric Research\\
Earth Observing Laboratory\\
Research Aviation Facility

\vfill{}
\cleardoublepage{} \pagenumbering{roman}

\renewcommand{\contentsname}{Table of Contents} \tableofcontents{}
\vfill{}
\eject

%% LyX 2.2.2 created this file.  For more info, see http://www.lyx.org/.

%\phantomsection \addcontentsline{toc}{section}{List of Figures}

\listoffigures

\clearpage %\phantomsection \addcontentsline{toc}{section}{List of Tables}

\listoftables

\clearpage{}

\renewcommand{\abstractname}{Preface and Abstract}\thispagestyle{plain}\begin{abstract}

This is Part 1 of a three-part series addressing errors that are present
in most archived measurements of temperature made from research aircraft.
In this first part, the time response of standard airborne thermometers
is quantified by determining the transfer function where the known
input is turbulent dynamic heating. Differential equations for the
response are proposed, and it is verified that the solutions to those
differential equations provide good representations of the response
characteristics in terms of fitted empirical coefficients. Those solutions
then can be used to correct the measurements to compensate in part
for the response characteristics of the sensors. Part 2 will use these
results to demonstrate that standard data processing used by most
operators of research aircraft introduces errors that arise from incorrect
adjustment for dynamic heating. Part 3 will use the results from the
first two papers to assess how these errors affect measurements of
the flux of sensible heat and will propose and evaluate a method for
correcting those measurements that removes a significant error in
that measured flux.

\end{abstract}

\clearpage{}

\section*{Acknowledgements}

\label{sec:acknowledgements}

This material is based upon work supported by the National Center
for Atmospheric Research, which is a major facility sponsored by the
National Science Foundation under Cooperative Agreement No. 1852977.
Any opinions, findings and conclusions or recom-mendations expressed
in this publication are those of the author(s) and do not necessarily
reflect the views of the National Science Foundation. Measurements
used here (\citet{VOCALS2011}, \citet{SOCRATES2019}, \citet{WECAN2018})
were collected in research projects (\citet{wood2011vamos}, \citet{albrecht2019cloud},
\citet{mcfarquhar2014southern}) that used the NSF/NCAR research aircraft.
Project descriptions and additional information can be found at \href{https://www.eol.ucar.edu/field_projects/}{this URL}.
The referenced project teams conducted the experiments, with flight
operations, data acquisition and processing, and other project support
by the Research Aviation Facility, Earth Observing Laboratory, National
Center for Atmospheric Research (NCAR). The analyses reported here
were mostly performed using R (\citet{Rlanguage}), with \index{RStudio}RStudio
(\citet{RStudio2012}) and \index{knitr}knitr (\citet{Xie2014a,Xie2014b}).
Data files in netCDF format have been read and written using the R
\index{R language!package!ncdf4}package ``ncdf4''; cf.~\citet{ncdf4}.
Substantial use also was made of the \index{ggplot2}\index{R language!package!ggplot2}``ggplot2''
package (\citet{wickham2009}) for R, and extensive use was made of
the ``stats'' package, part of Core R\@. Some of the numerical
integrations used the Runge-Kutta function from the ``rmutil'' package
(\citet{runge.kutta}).

\thispagestyle{plain}\clearpage{}

\thispagestyle{empty}

\cleardoublepage{}

\pagenumbering{arabic}

\section{Introduction}

Research aircraft routinely measure the air temperature, but the standard
sensors do not respond fast enough to meet many scientific needs.
In particular, measurements of the flux of sensible heat need faster
response than is typically available, as do measurements of near-discontinuous
changes such as those at the top of boundary layers or at cloud boundaries.
The measurement of sensible-heat flux requires, for the standard eddy-correlation
measurement, that temperature be measured with sufficient response
to resolve the spectrum of contributions to the flux. Various recent
reviews of priorities for research in atmospheric science have called
attention to the important roles that fluxes of various quantities
play in climate science and have advocated increased focus on those
fluxes; e.g., \citet{NAP6021}.

The basis for the measurement of the flux of sensible heat ($F_{s})$
by eddy correlation is this equation:

\begin{equation}
F_{s}=\rho_{a}\thinspace C_{p}\left\langle w^{\prime}T^{\prime}\right\rangle \label{eq:heatFlux}
\end{equation}
where $\rho_{a}$ is the density of air, $C_{p}$ the specific heat
of air at constant pressure, $w$ the vertical wind, and $T$ the
temperature. Primes in this equation denote fluctuations from the
mean and angle brackets denote an ensemble average. The measurement
thus depends on having a temperature sensor that can respond to the
range of fluctuations making significant contributions to the heat
flux. \citet{FrieheKhelif1992} suggested that 4--5~Hz is ``just
adequate'' (for flight at around 125~m/s) and that 25~Hz would
be desirable to resolve some interesting aspects of the temperature
structure. If the response of the temperature sensor is reduced or
shifted in phase at a particular frequency, an error will be introduced
into the measurement of sensible-heat flux. \citet{LawsonRodi1992}
argued that sensible-heat flux measured by some of the fastest sensors
then in common use produced measurements of sensible heat flux about
21\% too low compared to the measurements from their faster thermocouple-based
sensor. To avoid significant errors in this measurement, it therefore
is essential to characterize the time response of the temperature
sensor used and, where necessary, to apply corrections to compensate
for that response.

In this first paper, the time response of some standard airborne temperature
sensors is characterized in terms of a transfer function that relates
the measurand (the recovery temperature) to the measurement (the sensor
output) in ways that are invertible. Two coupled differential equations
are used as the basis for this characterization, but the transfer
function is determined independent of those equations. Because the
equations predict a transfer function matching the observations, they
provide a useful generalization when the measurements are fitted to
a three-parameter equation consistent with those equations. The measured
transfer functions for standard airborne thermometers then are used
in the papers that follow to access how common measurements are affected.





\section{Theory}

The errors considered here apply to the measurement from the sensor,
the ``recovery temperature,'' rather than the final temperature
after correction for dynamic heating. Because the sensor cannot respond
to rapid fluctuations, the standard correction for dynamic heating
introduces errors into the measured air temperature that are then
amplified by a correction procedure. This will be addressed in Part
2, but for this reason the present paper will discuss only the recovery
temperature.

\citet{PayneEtAl1994} provided a particularly illuminating analysis
of the expected response of a temperature sensor and developed their
results in terms of a transfer function. They represent the response
of the sensor in terms of two coupled differential equations, one
representing the temperature of the sensing wire and a second representing
the temperature of the structure that supports that wire. Their analysis
in terms of fundamental representation of the heat transport leads
to reasonable consistency with previous two-time-constant models like
that of \citet{mccarthy1973method} but poor agreement with the empirical
evidence regarding the time-response parameters in those models. They
note, however, that the empirical evidence is not as consistent or
convincing as would be desirable. One goal of the present work is
to develop a different and readily available method of determining
those parameters. The over-arching goal is to be able to assess errors
that might be present in measurements of the flux of sensible heat
and to apply corrections for those errors. This will be the subject
of Part 3.

\subsection{The differential equations and their solution}

Previous studies have demonstrated that a simple first-order exponential
equation with one time constant does not represent the time response
of airborne temperature sensors. The suggested explanation (\citet{NCAR_OpenSky_TECH-NOTE-000-000-000-064}
is that heat is transferred to the sensing wire of standard sensors
not only from the air but also from the supporting structure that
is in contact with the wire. Friehe and Khelif (\citet{FrieheKhelif1992}),
following other prior work including that of \citet{rodi1972analysis}
and \citet{mccarthy1973method}, suggested representing the two-time-constant
response via the following functional form:\\
\begin{equation}
\Theta(t)=A_{1}e^{-t/\tau_{1}}+A_{2}e^{-t/\tau_{2}}\label{eq:FrieheKehlif}
\end{equation}
where $\Theta(t)$ is the normalized history of the measured temperature
decaying from an initial value of unity to a final value of zero.
The sum of the coefficients $A_{1}$ and $A_{2}$ must then be 1.
The values for \{$A_{1},\,A_{2},\,\tau_{1},\,\tau_{2}$\} suggested
by \citet{FrieheKhelif1992} were \{0.65, 0.35, 0.09~s, 0.5~s\}.

Following the approach of \citet{PayneEtAl1994}, the time response
of the sensor will be represented by two coupled differential equations,
one that describes the response of the support on which the sensing
wire is wound to the air temperature and a second that describes the
response of the sensing wire to two inputs, one from the support and
one from the air. No attempt is made here to determine the parameters
from first principles as in \citet{PayneEtAl1994}, however; instead,
parameters entering the equations are determined empirically. The
equations are:\\
\begin{equation}
\frac{dT_{s}(t)}{dt}=\frac{T_{r}(t)-T_{s}(t)}{\tau_{2}}\label{eq:Ts}
\end{equation}
\begin{align}
\frac{dT_{m}(t)}{dt} & =\frac{a(T_{r}(t)-T_{m}(t))+(1-a)(T_{s}(t)-T_{m}(t))}{\tau_{1}}\label{eq:Tm}\\
= & \frac{\left\{ aT_{r}(t)+(1-a)T_{s}(t)\right\} -T_{m}(t)}{\tau_{1}}\nonumber 
\end{align}
where $T_{s}(t)$ is the temperature of the \uline{s}upport, $T_{m}(t)$
the \uline{m}easured temperature of the sensing wire, and $T_{r}(t)$
the true \uline{r}ecovery temperature that is the measurand. For
heat transfer to or from the wire, the parameter $a$ then represents
the fraction of the heat transferred by the air, while $(1-a)$ is
transferred to or from the support. The wire responds to the combined
transfers of heat with characteristic time constant $\tau_{1}$ while
the support structure responds to the air temperature more slowly,
with time constant $\tau_{2}$. It is straightforward to apply \eqref{eq:Ts}
and \eqref{eq:Tm} to changing but not necessarily discrete conditions,
so a general response to a given air-temperature history can be predicted
by numerical integration of these equations. Furthermore, the equations
are linear and, for constant values of the parameters, they are also
time-invariant (i.e., ``LTI'') descriptions of the response. As
a result, a particular signal for $T_{r}(t)$ can be decomposed into
its sinusoidal Fourier components and each will satisfy these equations
independently. The first equation does not involve the measurement,
so for a particular history of recovery temperature $T_{r}(t)$ the
support temperature can be determined solely by integration of \eqref{eq:Ts}.
Then, with $T_{s}(t)$ determined, \eqref{eq:Tm} can be integrated
to find the expected measurement $T_{m}(t)$ for a specified measurand
history $T_{r}(t)$. The inverse process, finding $T_{r}(t)$ from
the measurements $T_{m}(t)$, is also straightforward and only slightly
more complicated, as discussed in Appendix~\ref{sec:Correcting-the-Temperature}.

For a sinusoidal input these equations have analytic solutions after
any transient response from initial conditions has decayed. If the
actual recovery temperature is $T_{r}(t)=\sin\omega t$ where $\omega$
is the angular frequency, then the solutions for $T_{s}(t)$ and $T_{m}(t)$
are given by the following equations:

\begin{equation}
T_{s}(t)=b\sin(\omega t+\zeta)\label{eq:TsSolved}
\end{equation}
\begin{equation}
T_{m}(t)=c\sin(\omega t+\phi)=C_{1}\cos\omega t+C_{2}\sin\omega t\label{eq:TmSolved}
\end{equation}
\\
where\\
\[
b=\frac{1}{\sqrt{1+\omega^{2}\tau_{2}^{2}}}
\]
\[
\zeta=-\arctan(\omega\tau_{2})
\]
\begin{align*}
C_{1} & =\left(\frac{1}{1+\omega^{2}\tau_{1}^{2}}\right)\left(-\omega\tau_{1}\left(a+(1-a)b\cos\zeta\right)+(1-a)b\sin\zeta\right)\\
= & \frac{-\omega}{(1+\omega^{2}\tau_{1}^{2})}\left(\tau_{1}a+\frac{(1-a)(\tau_{1}+\tau_{2})}{(1+\omega^{2}\tau_{2}^{2})}\right)
\end{align*}
\begin{align*}
C_{2} & =\left(\frac{1}{1+\omega^{2}\tau_{1}^{2}}\right)\left(a+(1-a)b\cos\zeta+\omega\tau_{1}(1-a)b\sin\zeta\right)\\
= & \left(\frac{1}{1+\omega^{2}\tau_{1}^{2}}\right)\left(a+\frac{(1-a)(1-\omega^{2}\tau_{1}\tau_{2})}{(1+\omega^{2}\tau_{2}^{2})}\right)
\end{align*}
\begin{equation}
c=\sqrt{C_{1}^{2}+C_{2}^{2}}\label{eq:responseAmp}
\end{equation}
\begin{equation}
\phi=\arctan(C_{1}/C_{2})\label{eq:responsePhase}
\end{equation}

\citet{mccarthy1973method} used the derivative of the step-function
response to find the impulse response function and, from its Fourier
transform, the sensor response function. That leads to the following
alternate expressions for $C_{1}$ and $C_{2}$:\\
\[
C_{1}=-\omega\left(\frac{A_{1}\tau_{1}}{1+\omega^{2}\tau_{1}^{2}}+\frac{A_{2}\tau_{2}}{1+\omega^{2}\tau_{2}^{2}}\right)
\]
\[
C_{2}=\left(\frac{A_{1}}{1+\omega^{2}\tau_{1}^{2}}+\frac{A_{2}}{1+\omega^{2}\tau_{2}^{2}}\right)
\]
With $A_{2}=(1-a)/(1-\tau_{1}/\tau_{2})$ and $A_{1}=1-A_{2},$ these
are equivalent to the expressions for the same coefficients given
above \eqref{eq:responseAmp}. This demonstrates that the equations
\eqref{eq:Ts} and \eqref{eq:Tm} are a representation of the response
equivalent to \eqref{eq:FrieheKehlif} and to the equations used by
\citet{mccarthy1973method} and \citet{InverarityJTech2000}, among
others.

\subsection{The transfer function\label{subsec:The-resulting-transfer}}

The transfer function $H(\omega)=c(\omega)e^{i\phi(\omega)}$ then
characterizes how the sensor will respond to a unit-amplitude sine
wave with angular frequency $\omega=2\pi\nu$ where $\nu$ is the
frequency. For a particular set of parameters ($a=$0.733,
$\tau_{1}=$0.0308~s, $\tau_{2}=$0.447~s),\footnote{These parameters are approximately representative of an unheated Rosemount
102E4AL sensor used on the NSF/NCAR C-130, as will be demonstrated
in Sect.~\eqref{subsec:The-unheated-Rosemount}.} the amplitude response and phase delay of the transfer function is
shown in Fig.~\ref{fig:LTsolution}. Similar plots of the amplitude
(but not the phase) have been shown by \citet{mccarthy1973method}
and \citet{nicholls1978measurements}. Modified transfer functions
for two small changes to these parameters are also shown to illustrate
the sensitivity of the solution to these parameters. This figure illustrates
that serious errors will enter estimates of the sensible heat flux
if temperature fluctuations at frequencies above 1~Hz make a significant
contribution to the flux. The contribution to the cospectrum of temperature
and vertical wind will be reduced by the product of the amplitude
and the cosine of the phase (Fig.~\ref{fig:FE}). Most of the 10-Hz
contribution is missed, but even at 1~Hz the error is about 28\%.

\begin{knitrout}
\definecolor{shadecolor}{rgb}{0.969, 0.969, 0.969}\color{fgcolor}\begin{figure}

{\centering \includegraphics[width=\maxwidth]{figure/LTsolution-1} 

}

\caption[The amplitude and phase for the frequency domain transfer function of the Rosemount 102E4AL temperature sensor]{The amplitude and phase for the frequency domain transfer function of the Rosemount 102E4AL temperature sensor. The parameters representing that sensor, labeled "best", are $a$=0.733, $\tau_1=0.0308$ s and $\tau_2=0.447$ s. To illustrate sensitivity,  the curves labeled "t1=0.05" and "t2=0.6" use instead $\tau_1=0.05$ s and $\tau_2=0.6$ s, respectively.}\label{fig:LTsolution}
\end{figure}


\end{knitrout}

\begin{knitrout}
\definecolor{shadecolor}{rgb}{0.969, 0.969, 0.969}\color{fgcolor}\begin{figure}

{\centering \includegraphics[width=\maxwidth]{figure/FE-1} 

}

\caption[Fractional error in the cospectrum of sensible-heat flux caused by the delayed response of the Rosemount 102E4AL temperature sensor]{Fractional error in the cospectrum of sensible-heat flux caused by the delayed response of the Rosemount 102E4AL temperature sensor.}\label{fig:FE}
\end{figure}


\end{knitrout}

These equations and their solution provide a basis for correcting
either the measured temperature or the sensible-heat flux calculated
from the cospectrum in \eqref{eq:heatFlux}. Corrected values can
be obtained by several methods including integration of the equations
for the derivatives or by dividing the Fourier transform of the time
series by the transfer function and then using inverse Fourier transformation
to recover the corrected time series. Those correction schemes are
discussed in detail in Appendix A. To support such corrections, the
next section determines the transfer function experimentally. 

\section{Determining the Tranfer Function}

The evaluation of the time response that follows relies on the dynamic
heating produced by airspeed fluctuations. In steady conditions a
temperature sensor exposed to the air stream will measure the recovery
temperature, defined as the ambient temperature increased by the effect
of dynamic heating. Dynamic heating fluctuates as the airspeed fluctuates,
so in a turbulent wind field fluctuations with a measurable frequency
spectrum are imposed on the sensor. These fluctuations are often significantly
larger than real fluctuations in the ambient temperature. Dynamic
heating of temperature sensors is discussed for example by \citet{BangeEtAl2013.ch2}
(cf.~their Eq.~2.23), who expresses dynamic heating $Q$ as\\
\begin{equation}
Q=\alpha_{r}\frac{V^{2}}{2C_{p}}=T_{r}\left(\frac{\alpha_{r}M^{2}R_{a}/(2C_{v})}{1+\alpha_{r}M^{2}R_{a}/(2C_{v})}\right)\label{eq:DHterm}
\end{equation}
where $\alpha_{r}$ is the ``recovery factor'' characterizing the
extent to which the air is brought to rest relative to the sensor,
$V$ is the airspeed, $C_{p}$ and $C_{v}$ are respectively the specific
heat of air at constant pressure and constant volume. $T_{r}$ is
the (true) recovery temperature expressed in absolute units, $M$
the Mach number, and $R_{a}$ the gas constant for air. The ambient
air temperature $T_{a}$ is related to the recovery temperature and
the dynamic heating via\\
\begin{equation}
T_{r}=T_{a}+Q\,\,\,.\label{eq:recoveryTemperature}
\end{equation}
Because dynamic heating can exceed $20^{\circ}\mathrm{C}$ at jet-aircraft
flight speeds, it is often the dominant cause of fluctuations in the
recovery temperature. If the fluctuations in dynamic heating are higher
in frequency than those to which the sensor can respond, corresponding
fluctuations will be attenuated in the measured spectrum and the phase
of the measured response relative to the imposed signal will vary,
from near $0^{\circ}$ for fluctuations slow compared to sensor response
to near $90^{\circ}$ or even more\footnote{A sensor with a first-order time constant cannot produce a phase lag
of more than $90^{\circ}$, but larger lags are possible for systems
characterized by two time constants, as developed below.} for fluctuations fast compared to that response. The amplitude and
phase of the recovery temperature relative to the dynamic-heating
forcing therefore can be used as sensitive indicators of the response
characteristics of the sensor and can constrain parameters like $a$,
$\tau_{1}$ and $\tau_{2}$ that fit the predictions to the observations.
The evaluation in terms of the amplitude ratio and phase shift of
the recovery temperature in response to dynamic heating will be used
to characterize the transfer function and to determine if it is represented
adequately by the parameterized form given by \eqref{eq:responseAmp}
and \eqref{eq:responsePhase}.

\subsection{Measurements}

The present investigation uses measurements from two NSF/NCAR (National
Science Foundation / National Center for Atmospheric Research) research
aircraft, a Gulfstream V (hereafter, GV) and a Hercules C-130. The
temperature sensors producing the measurements are in widespread use
so these results should have broad applicability. Some aspects of
the uncertainty limits associated with these measurements of temperature
are included in an NCAR Technical Note (\citet{Cooper2016ncartn}),
which focused on the measurements of wind from the GV\@. That document
included an estimate that the standard uncertainty in measurements
of temperature from the GV is about $0.3^{\circ}\mathrm{C}$ and referenced
\citet{CooperEtAl2014} for supporting evidence. This limit applies
when the temperature being measured is varying slowly but does not
apply when the temperature changes rapidly. It is well known, however,
that temperature sensors in common use on research aircraft have time-response
characteristics that can affect the measurements. \citet{FrieheKhelif1992}
and \citet{LawsonRodi1992}, among many others, provide reviews of
the evidence for delayed response of the standard sensors. In particular,
the unheated Rosemount 102E4AL sensor has been used widely as a fast-responding
sensor, so it will be a focus of this three-part study.

This research uses data archives produced by three research projects,
the VOCALS (VAMOS Ocean-Cloud-Atmosphere-Land Study), CSET (Cloud
Systems Evolution in the Trades) and SOCRATES (Southern Ocean Clouds,
Radiation, Aerosol Transport Experimental Study) experiments. The
field projects are described by \citet{wood2011vamos}, \citet{albrecht2019cloud}
and \citet{mcfarquhar2014southern}, respectively. All included low-level
flight segments over the Pacific Ocean that are used in this paper.
The reference list includes appropriate DOI references to the measurements.

\subsection{The response to dynamic heating}

Because the airspeed $V$ is itself conventionally determined using
the processed air temperature $T_{a}$, via $V=M\sqrt{\gamma R_{a}T_{a}}$
where $\gamma=C_{p}/C_{v}$, the second expression in \eqref{eq:DHterm}
provides the advantage that it does not rely on prior calculation
of the air temperature $T_{a}$ but can be calculated from only the
recovery temperature $T_{r}$ and the Mach number. The Mach number
in turn depends only on measurements of the dynamic and ambient pressures,
with a small adjustment for the water vapor pressure. However, the
available measurement is not the true recovery temperature $T_{r}$
but instead the measured temperature $T_{m}$ which may not include
high-frequency fluctuations in $T_{r}$. This in turn affects the
estimated fluctuations determined from \eqref{eq:DHterm}. To minimize
this problem, regions were sought where the fluctuations in dynamic
heating were the dominant cause of fluctuations in recovery temperature.
Temporarily consider these approximations: $\alpha_{r}\approx1$,
$R_{a}/(2C_{v})\approx1/5$, and $M$ small enough that the denominator
of the right side of \eqref{eq:DHterm} can be assumed equal to unity.
Dynamic heating then is approximately $Q\approx T_{r}M^{2}/5$ and
fluctuations in $Q$ are related to those in $T_{r}$ and $M$ according
to\\
\begin{equation}
\frac{\delta Q}{Q}\approx\frac{\delta T_{r}}{T_{r}}+\frac{2}{5}\frac{\delta M}{M}\label{eq:QprimeOverQ}
\end{equation}
Because the measured recovery temperature $T_{m}$ may not include
true high-frequency fluctuations in $T_{r}$, the measured phase and
amplitude of the response to the dynamic-heating term may be distorted
from the correct value at frequencies where $\delta T_{m}/T_{m}$
differs from $\delta T_{r}/T_{r}$. In regions where the last term
in \eqref{eq:QprimeOverQ} dominates, underestimation of the fluctuations
in the recovery temperature arising from sensor response will cause
less significant errors in the measured fluctuations in dynamic heating
$Q$, and those errors can be addressed by correction procedures.

Figure~\ref{fig:Qprime} shows the contributions to the dynamic-heating
term from the two terms on the right side of \eqref{eq:QprimeOverQ},
except that $T_{m}$ is used instead of the unknown $T_{r}$. This
is based on a low-level flight segment with moderate turbulence where
the airspeed fluctuations were approximately consistent with an eddy
dissipation rate of $3\times10^{-4}\,\mathrm{m}^{2}\mathrm{s}^{-3}$.
The variance of the second term is more than 100 times that of the
first, indicating that the fluctuations in the first term are less
than 10\% of those in the second term. Therefore the right side of
\eqref{eq:DHterm} with $T_{m}$ in place of $T_{r}$ was used initially
to represent dynamic heating. Once a set of parameters was determined,
$T_{r}(t)$ was calculated using the first correction procedure discussed
in Appendix~\ref{sec:Correcting-the-Temperature}. Iteration using
this estimate of $T_{r}(t)$ in place of $T_{m}(t)$ led to a small
change in the fitted values of the parameters, and the estimate became
stable after only one iteration.



\begin{knitrout}
\definecolor{shadecolor}{rgb}{0.969, 0.969, 0.969}\color{fgcolor}\begin{figure}[t]

{\centering \includegraphics[width=\maxwidth]{figure/Qprime-1} 

}

\caption[Frequency-weighted spectral variance for $\delta T_m/T_m$ and 0.4$\delta M/M$ as functions of frequency ($\nu$) for a low-level flight segment from VOCALS flight 3, 21 Oct 2008 11:39:00 -- 11:52:00 UTC]{Frequency-weighted spectral variance for $\delta T_m/T_m$ and 0.4$\delta M/M$ as functions of frequency ($\nu$) for a low-level flight segment from VOCALS flight 3, 21 Oct 2008 11:39:00 -- 11:52:00 UTC. The wavelength scale shows the correspondence between frequency and wavelength at the average airspeed. The two terms are labeled "Tm" and "Mach" in the legend.}\label{fig:Qprime}
\end{figure}


\end{knitrout}



\subsubsection{The unheated Rosemount 102E4AL sensor\label{subsec:The-unheated-Rosemount}}

\begin{table}[H]
\begin{centering}
\begin{tabular}{|c|c|c|}
\hline 
Segment &
start &
end\tabularnewline
\hline 
\hline 
1 &
6:50:00 &
7:00:00\tabularnewline
\hline 
2 &
7:33:00 &
7:43:00\tabularnewline
\hline 
3 &
10:46:00 &
10:56:00\tabularnewline
\hline 
4 &
11:42:00 &
11:52:00\tabularnewline
\hline 
5 &
12:43:00 &
12:53:00\tabularnewline
\hline 
6 &
13:30:00 &
13:40:00\tabularnewline
\hline 
\end{tabular}
\par\end{centering}
\caption{Flight segments from flight 3 of the VOCALS project, 21 October 2008.
Listed times are UTC.}
\end{table}

To characterize the response of the Rosemount 102E4AL sensor, six
ten-minute low level flight segments in the marine boundary layer
from one flight of the NCAR/NSF C-130 in the ``VOCALS'' project
(\citet{wood2011vamos}), which studied low-level clouds over the
Pacific Ocean near Chile, were selected that had simular flight conditions
including the intensity of the turbulence. The time intervals are
listed in Table 1. For each flight segment, the phase and amplitude
ratio between the measurement and the dynamic heating term were calculated,\footnote{The R routine ``spec.pgram()'' was used with 25-point modified Daniell
smoothing.} and the results for all six segments were averaged in 200
logarithmically spaced intervals in frequency. The results for the
average phase are shown in Fig.~\ref{fig:Vphase}. The theoretical
curve is based on best-fit parameters as determined from these measurements
and those of the amplitude ratio, discussed next.

\begin{knitrout}
\definecolor{shadecolor}{rgb}{0.969, 0.969, 0.969}\color{fgcolor}\begin{figure}

{\centering \includegraphics[width=\maxwidth]{figure/Vphase-1} 

}

\caption[Phase lag of measured recovery temperature behind dynamic heating, for the measurements (with error bars) and for the theoretical response for the best-fit parameters (green line)]{Phase lag of measured recovery temperature behind dynamic heating, for the measurements (with error bars) and for the theoretical response for the best-fit parameters (green line). The error bars indicate two-standard-deviation ranges in the mean at each plotted point. Data from the  flight segments listed in Table 1.}\label{fig:Vphase}
\end{figure}


\end{knitrout}

\begin{knitrout}
\definecolor{shadecolor}{rgb}{0.969, 0.969, 0.969}\color{fgcolor}\begin{figure}[t]

{\centering \includegraphics[width=\maxwidth]{figure/checkAmplitude-1} 

}

\caption[The ratio of the spectral amplitude for the measurement of recovery temperature ($T_m(t)$) to that for dynamic heating ($Q$), shown as the plotted data points]{The ratio of the spectral amplitude for the measurement of recovery temperature ($T_m(t)$) to that for dynamic heating ($Q$), shown as the plotted data points. There are additional data points at frequencies below about 0.04 Hz that do not appear in this plot because they lie above the upper limit for the ordinate. The green line is the prediction from the transfer function determined from the best-fit values matching the phase lag between these variables, and the dashed orange line is a similar result with the second time constant $\tau_2$ increased from 0.447 to 0.6 s to illustrate sensitivity to this parameter.}\label{fig:checkAmplitude}
\end{figure}


\end{knitrout}

The ratio of the amplitude of the response to that of the dynamic-heating
signal, used as an estimate of the gain of the transfer function,
is shown in Fig.~\ref{fig:checkAmplitude}. It is useful to consider
both the amplitude and phase when determining the response parameters
because, as shown in Fig.~\ref{fig:LTsolution}, the amplitude of
the transfer function is more sensitive to $\tau_{2}$ than the phase
but $\tau_{1}$ is a very sensitive predictor of the phase at high
frequency. For the set of favored parameters, Fig.~\ref{fig:checkAmplitude}
shows the standard prediction and another with $\tau_{2}$ set to
0.6~s instead, to show the sensitivity of this result to that parameter.
The best prediction based on the measured phases consistently underestimates
the ratio of spectra for frequencies below about 0.1~Hz and above
about 3~Hz but is reasonably consistent with the observed ratio between
0.1~Hz and 3~Hz. Below 0.1~Hz it appears likely that the sensor
is responding to real fluctuations in temperature not attributable
to dynamic heating, as would be expected at these low frequencies.
Above 3~Hz the prediction is much too low, probably because there
is noise or other spurious variance in $T_{m}(t)$ not caused by dynamic
heating. 



The fit procedure used \eqref{eq:responseAmp} and \eqref{eq:responsePhase}
to find the theoretical value of the amplitude ratio and phase at
each frequency represented in the observations. For assumed values
of the three parameters $a$, $\tau_{1}$ and $\tau_{2}$, a chi-square
was calculated from the differences between these theoretical values
and the observed values. The frequencies used for the fit were 0.01
to 12~Hz for the measurements of phase and 0.1 to 3~Hz for the measurements
of amplitude ratio, to avoid regions where effects other than dynamic
heating appear to bias the measurements. Then a search procedure varied
these parameters to seek the minimum value of the chi-square.\footnote{The code can be found in the ``Rnw'' document that generates the
present document. It used the ``optim()'' function from the R ``stats''
package produced by the \citet{Rlanguage}.} The resulting values were $a=$ 0.73, $\tau_{1}=$
0.031 and $\tau_{2}=$ 0.45.
The chi-square for the fit is about 18 times larger than expected
if the fit represents the measurements to measurement uncertainty,
so it is difficult to assign uncertainty limits to this result on
the basis of this fit because of this not-understood excess chi-square,
but the fit minimum distinguished nearby values to about three significant
digits in all three parameters. The Hessian from the fit implies that
the results with standard uncertainties are $a=$ 0.733$\pm$0.004,
$\tau_{1}=$ 0.0308$\pm$\ensuremath{3\times 10^{-4}}
and $\tau_{2}=$ 0.45$\pm$0.02.



To complete the iteration discussed earlier, the measured recovery
temperature was then corrected via method 1 from Appendix~A, using
the parameters from this first fit, to find a prediction for the actual
recovery temperature $T_{r}(t)$. After recalculating $Q$ using \eqref{eq:DHterm}
with that estimate of $T_{r}(t)$ in place of $T_{m}(t)$, the calculation
of phase and amplitude was repeated and the results were fitted again
by adjusting the fit parameters. Only very minor changes arose from
this procedure even after one iteration, but the iterated result is
the one used here to represent the unheated Rosemount 102E4AL sensor.





The airflow and typical flow angles approaching a sensor can affect
its response, so the results might change when installed on a different
location or a different aircraft. Therefore a similar evaluation examined
the response of this same sensor when flown on the NSF/NCAR GV, which
flies significantly faster than the C-130. The results of a study
using a combined low-level dataset from the SOCRATES (\citet{mcfarquhar2014southern})
and CSET (\citet{albrecht2019cloud}) experiments, which were flown
over the Pacific Ocean, were similar to but slightly different from
the coefficients determined on the C-130, with both time constants
a little smaller than found for the C-130 ($\tau_{1}=$0.024
and $\tau_{2}=$0.22). This might be expected
at greater airspeed, as discussed in Sect.~\ref{subsec:Expected-dependence-on}.


\subsubsection{Heated sensors\label{subsec:Heated-sensors}}



Measurements from two slower sensors, a heated Goodrich/Rosemount
102 sensor and a similar \href{https://www.eol.ucar.edu/instruments/heated-ambient-temperature-sensor}{\textquotedblleft Harco Model 100009-1 Deiced TAT\textquotedblright}
(HARCO) sensor, have also been evaluated, but only the latter is included
here because they have similar response. The spectral variance for
both these measurements has apparent rapid attenuation beginning at
about 0.1~Hz, as shown in Fig.~\ref{fig:HARCOSpec}, and the response
is attenuated seriously above about 1~Hz.

Attempts to use the same three-parameter representation of the transfer
function relative to dynamic heating led to unsatisfactory fits, so
a different approach is used here. Because the evaluation in Sect.~\ref{subsec:The-unheated-Rosemount}
provides a good representation of the unheated Rosemount 102E4AL sensor,
the measurements from that sensor, corrected as will be described
in Appendix A, were used as the reference for the assumed-correct
recovery temperature. Then the phase and amplitude ratio were found
for the transfer function required to produce the heated-probe measurements
from the unheated-probe measurements. This did not require any assumptions
about equations or parameters determining the transfer function.







\begin{knitrout}
\definecolor{shadecolor}{rgb}{0.969, 0.969, 0.969}\color{fgcolor}\begin{figure}[t]

{\centering \includegraphics[width=\maxwidth]{figure/HARCOSpec-1} 

}

\caption[Spectral variance $P(\nu)$ weighted by frequency ($\nu$) for the recovery temperature measured by a heated  HARCO and an unheated Rosemount sensor]{Spectral variance $P(\nu)$ weighted by frequency ($\nu$) for the recovery temperature measured by a heated  HARCO and an unheated Rosemount sensor.}\label{fig:HARCOSpec}
\end{figure}


\end{knitrout}



To characterize the response of the heated HARCO sensor, boundary-layer
flight segments from the SOCRATES and CSET projects (referenced earlier
in connection with the unheated probe) were compiled into one data
set from the flight periods shown in Table~2. An unheated Rosemount
102E4AL sensor was also available, so corrected measurements from
that sensor were used as the reference against which to determine
the gain and phase of the transfer function.
\begin{table}[h]
\begin{centering}
\begin{tabular}{|c|c|c|}
\hline 
\textbf{Project / Flight} &
\textbf{start {[}UTC{]}} &
\textbf{end {[}UTC{]}}\tabularnewline
\hline 
\hline 
CSET / 5 &
2015-07-14 17:52:00 &
18:02:00\tabularnewline
\hline 
CSET / 5 &
2015-07-14 19:45:30 &
19:55:30\tabularnewline
\hline 
CSET / 5 &
2015-07-14 20:37:17 &
20:47:17\tabularnewline
\hline 
SOCRATES / 15 &
2018-02-24 5:52:00 &
6:02:00\tabularnewline
\hline 
SOCRATES / 15 &
2018-02-24 6:05:00 &
6:15:00\tabularnewline
\hline 
\end{tabular}
\par\end{centering}
\caption{Flight segments used to determine the response characteristics of
a heated HARCO sensor.}
\end{table}

\begin{knitrout}
\definecolor{shadecolor}{rgb}{0.969, 0.969, 0.969}\color{fgcolor}\begin{figure}

{\centering \includegraphics[width=\maxwidth]{figure/GVHARCO-1} 

}

\caption[The gain (top) and phase (bottom) for the transfer function characterizing a heated HARCO temperature sensor]{The gain (top) and phase (bottom) for the transfer function characterizing a heated HARCO temperature sensor. The measurements are indicated by error bars that show two-standard-deviation limits from the mean value). Two fits to the measurements, one based on the three-parameter representation ("3-par") and one on a polynomial fit ("lfit"), are described in the text.}\label{fig:GVHARCO}
\end{figure}


\end{knitrout}

The measured phase and amplitude ratio for this data set are shown
in Fig.~\ref{fig:GVHARCO}. The fit for the response function defined
by \eqref{eq:responseAmp} and \eqref{eq:responsePhase} is shown
as the blue line labeled ``3-par'' in that figure. The fitted values
for \{$a,\,\tau_{1},\,\tau_{2}$\} were \{0, 0.05,
1.12\}, and to obtain this result the fit
had to be constrained to keep $a$ non-negative. A value of zero for
the parameter $a$ would indicate that no heat is transferred from
the sensing wire to the air, but instead all is transferred to the
support which has a relatively slow characteristic response. 

The three-parameter fit is not consistent with the measurement errors
even though it provides an approximate representation of the transfer
function. The apparent reason is that there is conflict between the
constraints imposed by the amplitude ratio and the phase, such that
either could be represented reasonably but not both. The actual transfer
function has some complex features, including frequencies where the
phase shift reaches values below $-90^{\circ}$ (not possible for
a simple exponential time response) and values of the phase shift
of about $-38^{\circ}$ at 0.1~Hz where the amplitude ratio is still
high (about 0.75). The amplitude decreases to $e^{-1}$ at about 0.34~Hz,
as would be the case for a first-order time constant of about 1.3~s,
so this could be considered another measure of the response. However,
that value does not extrapolate well to other frequencies and the
phase shift at 0.34~Hz is approximately $-73^{\circ}$, which would
indicate that the measurement of a real contribution to sensible-heat
flux at this frequency would be only about 10\% of the correct value.

Because the three-parameter fit distorted the measured result, fits
in the logarithm of the frequency were used to provide a better representation
of the measurements, as shown by the dashed green lines labeled ``lfit''.
Those fits are given by these equations and coefficients, with $x=\log_{e}(\nu/\nu_{0})$
where $\nu$ is the frequency, $\omega=2\pi\nu$ and $\nu_{0}=1\,\mathrm{Hz}$:\\
\begin{equation}
\mathrm{for}\,\nu>0.024\,\mathrm{Hz,\,\,}H(\omega)=(h_{0}+h_{1}x+h_{2}x^{3}+h_{3}x^{4}+h_{4}x^{5})e^{i\phi(\omega)}\label{eq:lfitH}
\end{equation}
\[
\mathrm{for\:\nu\leq0.024\,\mathrm{Hz},\quad}H(\omega)=1
\]
\[
\phi(\omega)=p_{0}+p_{1}x+p_{2}x^{2}+p_{3}\arctan(\nu/\nu_{0})
\]
The coefficients obtained by fitting to the observations are $h_{0-4}=$\{$0.161$,
$-0.136$, $-0.0760$,
$-0.0309$, $-0.00324$\} and $p_{0-3}=$\{$-155.2$,
$-62.8$, $-6.35$, $67.2$\}.
This fit can be used to represent the transfer function better than
the three-parameter fit (with negative-frequency values defined as
the complex conjugate of the values at the corresponding positive
frequency), although the fit needs to be modified above about 2~Hz
because those values were not constrained by the measurements. A suggested
modification is to duplicate the value at 2~Hz to higher frequencies;
this appears to be adequate because there is so little variance measured
by this sensor at these frequencies, but it is important to avoid
possible zeroes that otherwise arise from extrapolation.


\subsubsection{Expected dependence on flight conditions\label{subsec:Expected-dependence-on}}

Based on measurements in a wind tunnel, \citet{GoodrichTR5755} indicated
that the fast-response characteristic time $\tau_{1}$ for the unheated
Rosemount 102E4AL sensor varies approximately as $\log(Z^{-0.6})$
where $Z=M\rho_{a}/\rho_{s}$ with $M$ the Mach number, $\rho_{a}$
the air density and $\rho_{0}$ the air density under standard conditions.
The mean value of $Z$ for the flight segments used to find the best-fit
parameters was $Z=0.3$, so this suggests that the first characteristic
time for that sensor is best represented by

\begin{equation}
\tau_{1}^{\prime}(Z)=\tau_{1}\left(\frac{0.3}{Z}\right)^{0.6}\,\,\,.\label{eq:tau1prime}
\end{equation}

 There is no similar evidence for $\tau_{2}$, but it might be expected
to have similar dependence because this is approximately the Reynolds
number dependence and the Nusselt number characterizing ventilated
heat transfer often is represented by a power-law relationship to
the Reynolds number. If both heat transfer terms scale similarly,
it might be expected that $a$ will be unchanged.

For these reasons, the time parameters obtained in preceding sections
have been adjusted to a reference value of $Z=0.3$ in Table~\ref{tab:Parameters}.
For other conditions, it is suggested that the best estimate will
be to multiply $\tau_{1}$ and $\tau_{2}$ by $(0.3/Z)^{0.6}$. (Having
these parameters vary is in conflict with the ``LTI'' assumption
leading to the transfer function, but these variations are minor over
short times so incorporating this variation should produce reasonable
results.)


\begin{center}
\begin{table}[h]
\begin{centering}
\begin{tabular}{|c|c|c|c|}
\hline 
\textbf{sensor} &
\textbf{$a$} &
\textbf{$\tau_{1}$} {[}s{]} &
\textbf{$\tau_{2}$} {[}s{]}\tabularnewline
\hline 
\hline 
unheated Rosemount 102E4ALon C-130 &
0.73 &
0.031 &
0.45\tabularnewline
\hline 
unheated Rosemount 102E4AL on GV &
0.65 &
0.024 &
0.22\tabularnewline
\hline 
heated HARCO &
0.0 &
0.058 &
1.29\tabularnewline
\hline 
\end{tabular}
\par\end{centering}
\centering{}\caption{Parameters for the time response of available temperature sensors
on the NSF/NCAR aircraft, adjusted to $Z=0.3$. For other conditions,
scale as represented for $\tau_{1}^{\prime}$ in \eqref{eq:tau1prime}.\label{tab:Parameters}}
\end{table}
\par\end{center}


\subsection{Response to a step change}



\begin{knitrout}
\definecolor{shadecolor}{rgb}{0.969, 0.969, 0.969}\color{fgcolor}\begin{figure}[t]

{\centering \includegraphics[width=\maxwidth]{figure/tfit-1} 

}

\caption[(blue dots)]{(blue dots): Temperature measured at 25 Hz during descent through an inversion capping the marine boundary layer, from VOCALS flight 3, starting at 8:13:50 UTC. The descent rate was approximately 5 m/s. The dashed red line shows a dry-adiabatic lapse rate in the marine boundary layer, and the dashed black line is a reference line indicating the location of the top of the boundary layer.The  prediction using the parameters listed in the text is shown as the magenta line, mostly over the blue dots representing the measurements. The dashed brown line is the calculated temperature of the support that contacts the sensing wire.}\label{fig:tfit}
\end{figure}


\end{knitrout}

The preceding subsection developed estimates of the sensor response
with low uncertainty compared to previous estimates, so that will
be the primary constraint on these parameters. It is nevertheless
useful to check if other measurements that are sensitive to the time
response are consistent with the characterization developed in the
preceding subsections. Previous studies have mostly used sharp temperature
changes in the atmosphere, for example from climbs through the inversion
at the top of a boundary layer, to study the time response. A search
of representative VOCALS climbs and descents through inversions capping
the marine boundary layer found many with unusable structure but one
near-ideal example, from VOCALS research flight 3, with a descent
through the inversion at 8:13:50 UTC. Figure \ref{fig:tfit} shows
the time history of the measured temperature for 8~s during this
descent, which was at a rate of approximately 1000~ft/min (around
5~m/s). The temperature structure in this case was remarkably consistent
with a near-constant temperature above the inversion and a near-adiabatic
temperature structure below the inversion.

The suggested measurand history, if the discontinuity at the inversion
is discrete, is that shown by the dashed black and dashed red lines..
The predicted time response from \eqref{eq:Tm} for assumed time constants
for the unheated Rosemount 102E4AL sensor adjusted for air density
and flight speed is shown as the magenta line in Fig.~\ref{fig:tfit}.
The predicted response is consistent with the observations and supports
the approximate validity of the parameters determined from fits to
the response to dynamic heating.

\subsection{Application to a ``speed run''}

Another situation where a temperature lag can be observed is when
the airspeed changes during level flight and causes a change in dynamic
heating. An example is shown in Fig.~\ref{fig:SRDW}, where the airspeed
was increased steadily in level flight from near the lower limit of
the flight envelope to near the upper limit and then was decreased
back to the starting value. A plot of recovery temperature should
also increase and decrease as the dynamic heating changes, but with
a lag caused by the sensor response. This lag will produce hysteresis
in the measured temperature during the speed run.

\begin{knitrout}
\definecolor{shadecolor}{rgb}{0.969, 0.969, 0.969}\color{fgcolor}\begin{figure}[t]

{\centering \includegraphics[width=\maxwidth]{figure/SRDW-1} 

}

\caption[History of the airspeed during a  "speed-run" maneuver where the airspeed varied during level flight over the available speed range of the aircraft]{History of the airspeed during a  "speed-run" maneuver where the airspeed varied during level flight over the available speed range of the aircraft.}\label{fig:SRDW}
\end{figure}


\end{knitrout}

Figure~\ref{fig:SRRTH1delay}a shows this hysteresis, which appears
as the difference between the segment with increasing speed and that
with decreasing speed. A correction based on simply shifting the
measurements in time works reasonably but doesn't take into account
that the Mach number and hence the time parameters vary significantly
during the speed run. A better test of the time-response parameters
is to apply the first correction scheme outlined in Appendix~\ref{subsec:The-resulting-transfer}
to the measurements, with varying response parameters dependent on
the Mach number. The specific correction equation used is \eqref{eq:HARCOsoln}.
Figure~\ref{fig:srRTH1delay}b shows that the delay is mostly removed
by this procedure. The residual standard deviation about the regression
fit for recovery temperature as a function of $V^{2}/(2C_{p})$ is
reduced from $0.26^{\circ}$C before correction to $0.10^{\circ}$C
after correction. The minimum standard deviation results from increasing
the time constants an additional 10\%, so measurements from this speed-run
maneuver are consistent with the predicted time response as found
in Sect.~\ref{subsec:Heated-sensors} to within about this uncertainty.



\begin{knitrout}
\definecolor{shadecolor}{rgb}{0.969, 0.969, 0.969}\color{fgcolor}\begin{figure}

{\centering \includegraphics[width=\maxwidth]{figure/srRTH1delay-1} 

}

\caption{The recovery temperature as measured by a heated HARCO sensor during the speed-run maneuver shown in the previous figure (a) and the same measurement after correction (b). The abscissa is the dynamic-heating term without the recovery factor, where $V$ is the airspeed and $Cp$ is the specific heat at constant pressure. The measurements while the airspeed was increasing are shown by the green lines and those for decreasing airspeed by the blue lines. The dashed orange lines indicate the regression fits, with standard deviation about the fit of 0.26$^{\circ}$C (uncorrected) and 0.10$^{\circ}$C (after correction).}\label{fig:srRTH1delay}
\end{figure}


\end{knitrout}



\clearpage{}

\section{Uses of the Transfer Functions}

The transfer functions determined in the preceding section have two
potential uses: (i) to predict how airborne thermometers will respond;
and (ii) to correct measurements to compensate for the time response
of the sensors. The two papers that follow, Part 2 and Part 3, are
examples of these two uses. In Part 2, the transfer functions are
used to assess how sensors respond to dynamic heating and to develop
appropriate correction schemes. Part 3 applies the transfer functions
to improve measurements of the flux of sensible heat.

It is also possible to make corrections to the measurements for other
purposes. Once the transfer functions are determined, the correction
procedures use standard techniques, as discussed in Appendix A. These
are suited to standard measurements made from most research aircraft
and so can have general community utility.

\section{Summary and Conclusions}

Findings and conclusions of this investigation include these items:
\begin{enumerate}
\item The differential equations \eqref{eq:Ts} and \eqref{eq:Tm}, with
appropriate parameters, provide an analytical representation of the
transfer function for the recovery temperature measured by an unheated
Rosemount 102E4AL sensor. That transfer function was shown to be consistent
with measurements of the phase and amplitude ratio of the response
to dynamic-heating fluctuations. This is good evidence that the equations
provide a good representation of the time response for that sensor.
The predictions of the equations are less satisfactory when applied
to a heated HARCO sensor or a heated Rosemount sensor, possibly indicating
that the heat transfer is not represented adequately by those equations.
\item For the Rosemount 102E4AL sensor, the three parameters in those equations
9characterizing the two time constants and the fraction of heat transfer
to the air vs.~that to the structure supporting the sensing wire)
can be determined with small uncertainty by fitting the transfer function
to observations of dynamic heating. These parameters are thus constrained
well and can be relied upon to make corrections to the measurements
and otherwise to characterize the effects of time response of that
sensor.
\item Some additional evidence supports the general magnitude of the parameters
determined by reference to dynamic heating. The supplemental observations
come from a sharp feature in the atmosphere and from the lag observed
during flight while the airspeed varies at constant altitude.
\item Once the transfer function for the unheated Rosemount sensor has been
determined, it can be used to estimate the true recovery temperature,
and then transfer functions for other sensors can be determined by
comparison to that estimate of the measurand to which they are responding.
This approach has been used here for the slower heated sensors and
should provide a means of correcting other sensors slower than the
unheated sensor. Appendix A uses these results with standard methods
to correct the measurements from airborne temperature sensors for
their time response.
\end{enumerate}

\appendix

\section{Correcting the Temperature\label{sec:Correcting-the-Temperature}}

The true recovery temperature $T_{r}$ can be retrieved from the measured
temperature $T_{m}$ in two ways, either from the differential equations
or by Fourier transformation. These methods are illustrated here.

The differential equations \eqref{eq:Ts} and \eqref{eq:Tm} provide
a basis for correcting the measured temperature to account for the
time response of the sensor. They should be applied to the recovery
temperature $T_{r}(t)$, which is the measurand. The actual measurement
is $T_{m}(t)$ and the temperature of the support is $T_{s}(t)$,
so those equations, rearranged, are:

\begin{equation}
\frac{dT_{s}(t)}{dt}=\frac{T_{r}(t)-T_{s}(t)}{\tau_{2}}\label{eq:Ts2}
\end{equation}
\begin{equation}
T_{r}(t)=\frac{1}{a}\left\{ \tau_{1}\frac{dT_{m}(t)}{dt}+T_{m}(T)-(1-a)T_{s}(t)\right\} \label{eq:Tm2}
\end{equation}

There are two unknowns ($T_{r}(t)$, the actual recovery temperature,
and $T_{s}(t)$). Those unknowns are specified by the two preceding
equations because all other terms are known, including $dT_{m}/dt$
which can be represented using the numerical derivative of the measurements
$T_{m}(t)$. The second equation can be used to eliminate $T_{r}$
from the first:\\
\begin{equation}
\frac{dT_{s}(t)}{dt}=\frac{\frac{1}{a}\left\{ \tau_{1}\frac{dT_{m}(t)}{dt}+T_{m}(T)-(1-a)T_{s}(t)\right\} -T_{s}(t)}{\tau_{2}}\label{eq:Ts3}
\end{equation}
From an initial value $T_{s}(0)$, assumed to be $T_{m}(0)$, this
equation can be integrated to find the temperature of the support,
$T_{s}(t)$. Once that is known, \eqref{eq:Tm2} specifies the estimate
of the true recovery temperature $T_{r}(t)$ without further integration.
The only choices needed are the numerical method used to find the
derivative $dT_{m}/dt$ (e.g., here centered fourth-order) and the
integration method applied to \eqref{eq:Ts3}, here fourth-order Runge-Kutta
integration with Cash-Karp (\citet{cash1990variable}) adjustment
of the step size. If a centered second-order finite-difference expression
is used for $dT_{m}(t)/dt$ and an Euler integration is used to integrate
\eqref{eq:Ts3}, this correction can be shown to be equivalent to
that developed by \citet{InverarityJTech2000}; cf.~his Eqn.~(12).





An example of the corrected recovery temperature will be shown in
the next subsection, where the two approaches to correction can be
compared.

An alternate approach is to use Fourier transforms:
\begin{enumerate}
\item Calculate the Fourier transform of the measured time series: $\hat{T}_{m}(\omega)=\mathcal{F}\left(T_{m}(t)\right)$
where $\omega$ is the angular frequency and $\mathcal{F}$ denotes
the Fourier transform.
\item Divide the result by the complex representation of the transfer function:
$\hat{T}_{r}(\omega)=\hat{T}_{m}(\omega)/H(\omega)$.
\item Use the inverse Fourier transform to find a retrieved estimate of
the true recovery temperature: $T_{r}(t)=\mathrm{Re}\left(\mathcal{F}^{-1}\left(\hat{T}_{r}(\omega)\right)\right)$
where Re denotes the real part of the complex result.
\end{enumerate}


\begin{knitrout}
\definecolor{shadecolor}{rgb}{0.969, 0.969, 0.969}\color{fgcolor}\begin{figure}

{\centering \includegraphics[width=\maxwidth]{figure/sampleFFT-1} 

}

\caption[Example of the changes produced by the correction procedures]{Example of the changes produced by the correction procedures. The original measurement of recovery temperature is Tm, produced by an unheated Rosemount sensor, and the revised values are RT (from integration) and FFT (from Fourier transforms). The dashed green line labeled "Ts" is the temperature of the support as calculated using (17). Measurements from VOCALS flight 3.}\label{fig:sampleFFT1}
\end{figure}

\begin{figure}

{\centering \includegraphics[width=\maxwidth]{figure/sampleFFT-2} 

}

\caption[Variance spectra for the original measurement of recovery temperature (Tm) produced by an unheated Rosemount sensor and for the corrected values (RT and FFT) produced respectively by (16) and the Fourier-transform algorithm]{Variance spectra for the original measurement of recovery temperature (Tm) produced by an unheated Rosemount sensor and for the corrected values (RT and FFT) produced respectively by (16) and the Fourier-transform algorithm.}\label{fig:sampleFFT2}
\end{figure}


\end{knitrout}

Sample results of these correction procedures are shown in Fig.~\ref{fig:sampleFFT1}.
The agreement between the two correction methods is very good, and
both show evidence of faster and higher-amplitude response to fluctuations.
The resulting variance spectrum for the Fourier-transform method (Fig.~\ref{fig:sampleFFT2})
has high variance above about 5~Hz that is above the variance for
the corrected variable obtained from \eqref{eq:Tm2}. The increasing
variance for the Fourier-transform method is is likely the fault of
the measurement itself: The spectral variance for the original measurement
(Tm) does not decrease at high frequency as expected from the transfer
function, and the correction procedure amplifies this excess noise.
The lower variance for the variable obtained from \eqref{eq:Tm2}
at high frequency arises from the finite-difference representation
of the term $dT_{m}(t)/dt$, which results in some smoothing.

Because the heated HARCO sensor is much slower than the unheated
Rosemount 102E4AL sensor, the measurements from that slower sensor
can't be corrected to the extent possible for the unheated sensor,
but it is still useful to evaluate to what extent the measurements
can be improved. The HARCO presents a special case because the best-fit
value is $a=0$ so \eqref{eq:Tm2} can't be used. However, in this
case the differential equations can still be combined to give\\
\begin{equation}
T_{r}(t)=(\tau_{1}+\tau_{2})\frac{dT_{m}(t)}{dt}+T_{m}(t)+\tau_{2}\tau_{1}\frac{d^{2}T_{m}(t)}{dt^{2}}\label{eq:HARCOsoln}
\end{equation}
a form that can be used directly without integration because finite-difference
expressions can be used for the derivatives of the measured $T_{m}(t)$.
The solution from \eqref{eq:HARCOsoln} is very noisy if finite-difference
estimates of the derivatives are used, so smoothing of the result
was applied. Figure~\ref{fig:HARCOcorrection1} shows the result
(as ``RTH'') after a Butterworth low-pass filter with cutoff frequency
of 2~Hz smoothed the corrected measurements. This filtering is reasonable
because the variance spectrum for this sensor shows very little real
signal at frequencies above 2~Hz. In comparison to the original measurement
(labeled ``Tm''), the response of the sensor is greatly improved
by this correction procedure. It even provides a reasonable representation
of the corrected unheated Rosemount measurement (labeled ``RT'')
for the same period.

This is not as good a representation of the transfer function as is
possible with the fitted representation shown in Fig.~\ref{fig:GVHARCO}
and given by \eqref{eq:lfitH}. That fit can be used with the Fourier-transform
approach to correction. The result is shown in Fig.~\ref{fig:HARCOcorrection1}
as the black line labeled ``FFT''. This is also a significant improvement
over the original and reproduces many of the features of the best
measurement (``RT''). To obtain this result, it was necessary to
attenuate frequencies above 1.8~Hz in the Fourier transform solution
because there is a zero in the transfer function as represented by
\eqref{eq:lfitH} that otherwise invalidates the inversion. This attenuation
was accomplished by multiplying the transfer function by $e^{5\nu}$above
1.8~Hz. This arbitrarily chosen attenuation gave reasonable results,
although it is likely that better choices could be made with further
exploration. One more general measure of the improvement, beyond the
anecdotal evidence in the preceding figures, is that either correction
procedure reduced the standard deviation of the difference between
the measured value (Tm) and the estimated best value (RT) from $0.06^{\circ}$C
before correction to $0.03^{\circ}$C after correction.

The plot of variance spectra (Fig.~\ref{fig:HARCOcorrection2}) shows
that the original spectrum (blue line labeled ``Tm'') is seriously
attenuated at high frequencies relative to the reference measurement
(``RT'') and that both correction procedures restore significant
parts of the missing spectral variance. Despite its inferior representation
of the transfer function, it appears that application of \ref{eq:HARCOsoln}
provides a better match to the reference measurements. It does not
appear possible to restore the missing high-frequency fluctuations
(above about 1 or 2~Hz) because the original measurement is so severely
attenuated at these frequencies. This sensor and the similar heated
Rosemount sensor are therefore unable to detect contributions to sensible-heat
flux from this frequency range, even after corrections. It nevertheless
appears useful to apply one of these correction approaches routinely
to improve the quality of this measurement.

\begin{knitrout}
\definecolor{shadecolor}{rgb}{0.969, 0.969, 0.969}\color{fgcolor}\begin{figure}

{\centering \includegraphics[width=\maxwidth]{figure/HARCOcorrection-1} 

}

\caption[Corrected recovery temperature as measured by a heated HARCO sensor ("RTH" and "FFT"), the uncorrected measurement ("Tm"), and the best estimate of the true recovery temperature ("RT") based on an unheated Rosemount sensor after correction]{Corrected recovery temperature as measured by a heated HARCO sensor ("RTH" and "FFT"), the uncorrected measurement ("Tm"), and the best estimate of the true recovery temperature ("RT") based on an unheated Rosemount sensor after correction. The time is seconds after 2018-02-24 5:59:00 UTC, SOCRATES flight 15. "RTH" is based on the approximate formula (20), while "FFT" results from Fourier transformation after correction using the transfer function determined from (13). Mean values have been subtracted from all to facilitate comparisons.}\label{fig:HARCOcorrection1}
\end{figure}

\begin{figure}

{\centering \includegraphics[width=\maxwidth]{figure/HARCOcorrection-2} 

}

\caption[Variance spectra for some measurements of recovery temperature]{Variance spectra for some measurements of recovery temperature: "RT" (the best estimate resulting from correction of the measurements from the unheated Rosemount 102E4AL); "Tm" (the uncorrected measurement from the heated HARCO); "FFT" (the corrected HARCO measurement based on Fourier transforms); and "RTH" (the corrected HARCO measurement based on the correction formula (17)).}\label{fig:HARCOcorrection2}
\end{figure}


\end{knitrout}

\clearpage{}

\section{Reproducibility}

\index{reproducibility}This document is constructed in ways that
support duplication of the study. The code that generates the plots
and implements the correction procedure is incorporated into the same
\index{program!file}file that generated this document via \LaTeX,
using principles and techniques described by \citet{Xie2014a} as
implemented in the R \index{R language!package!knitr}package\index{knitr}
'knitr' (\citet{Xie2014b}). The program, 'Paper1.Rnw', is archived
on 'GitHub' \index{GitHub repository}in the \index{repository!github}directory\index{archive!for this document}
at \href{https://github.com/WilliamCooper/KalmanFilter.git}{this URL}.
There is some \index{supplemental material}supplemental material
in that directory, including a workflow document\index{workflow document}
for all the papers, the bibliography and some code segments saved
in the ``chunks''\index{R language!program chunks} subdirectory,
so the full directory should be downloaded in order to run the program.
The calculations use the programming \index{R language}language \index{R language!program}\index{RStudio}R
(\citet{Rlanguage}) and were run within \index{RStudio}RStudio (\citet{RStudio2012}),
so this is the most straightforward way to replicate the calculations
and the generation of this \index{program!generating this document}document.

A \index{R language!package!Ranadu}package named Ranadu,\index{R language!package!Ranadu}
containing auxillary \index{function!Ranadu}functions, is used extensively
in the R code. It is available on GitHub\index{GitHub repository}
as \href{https://github.com/WilliamCooper/Ranadu.git}{https://github.com/WilliamCooper/Ranadu.git}.
The version used for calculations in this technical note is included
in the 'zip' archive listed below.

The \index{file!data}data files used are also preserved in the NCAR/EOL
Data Archives and can be obtained via a \index{data!requesting}request
to \url{mailto:raf-dm@eol.ucar.edu} or via the ``Data Access''
links at \href{https://www.eol.ucar.edu/all-field-projects-and-deployments}{this web site}.
The original files containing the data as produced by the NCAR Earth
Observing Laboratory, Research Aviation Facility, were in \index{netCDF format}netCDF
format (cf.~\href{http://www.unidata.ucar.edu/software/netcdf/}{this URL}),
but in many cases data archives were reprocessed and the files may
change after reprocessing so a separate archive\index{archive!for this document!data}
is maintained for this document. The data files\index{file!data!archive}
in this archive contain \index{R language!data.frames}R data.frames
and are preserved as binary-format 'Rdata' files via R 'save' commands.
The code in the GitHub archive has appropriate 'load' commands to
read these data files from a subdirectory named 'Data' (/Data or \textasciitilde /Data
or /home/Data) but this is not part of the GitHub repository because
it is too large to be appropriate there. To reproduce this research,
those data files have to be transferred separately from \{??where??\}

Extensive use has been made of \index{attributes!data.frame}\index{attributes!variable}attributes
assigned to the data.frames and the variables in those data.frames.
All the attributes from the original netCDF \index{file!netCDF}files
have been transferred to the data.frames, so there is a record of
how the original data were processed, for example with \index{calibration!coefficients!used in processing}calibration
coefficients and processing dependence for the variables. Key information
like the processing date, the program version that produced the archive,
and the selection of primary variables for various measurements thus
is preserved.

(See the related list of project components on the next page that
are preserved to enhance reproducibility.)

\clearpage{}

\begin{tabular}{ll}
\textsf{\textsc{\textcolor{blue}{Project:}}} &
SensibleHeatFlux\tabularnewline
\textsf{\textsc{\textcolor{blue}{Archive package:}}} &
\href{https://github.com/WilliamCooper/SensibleHeatFlux/blob/master/SensibleHeatFluxPaper1.zip}{SensibleHeatFluxPaper1.zip}\tabularnewline
\textsf{\textsc{\textcolor{blue}{Contains:}}} &
attachment list below\tabularnewline
\textsf{\textsc{\textcolor{blue}{Program:}}} &
\href{https://github.com/WilliamCooper/SensibleHeatFlux/blob/master/Paper1.Rnw}{Paper1.Rnw}\tabularnewline
\textsf{\textsc{\textcolor{blue}{Original Data:}}} &
\citet{VOCALS2011}, \tabularnewline
 &
\citet{CSET2017},\tabularnewline
 &
,\citet{SOCRATES2019}\tabularnewline
\textsf{\textsc{\textcolor{blue}{Special Data Files:}}} &
SensibleHeatFluxTechNote.Rdata, SensibleHeatFluxTechNote2.Rdata\tabularnewline
\textsf{\textsc{\textcolor{blue}{Workflow Document:}}} &
\href{https://github.com/WilliamCooper/SensibleHeatFlux/blob/master/WorkflowSensibleHeatFlux.pdf}{WorkflowSensibleHeatFlux.pdf}\tabularnewline
\textsf{\textsc{\textcolor{blue}{Git:}}} &
\index{GitHub repository}\href{https://github.com/WilliamCooper/SensibleHeatFlux.git}{https://github.com/WilliamCooper/SensibleHeatFlux.git}\tabularnewline
\end{tabular}

\attachm{Paper1.Rnw\\
Paper1.pdf\\
WorkflowSensibleHeatFlux.pdf\\
WAC.bib\\
chunks/{*}\\
SessionInfo}

\label{sec:bibliography}

\bibliographystyle{plainnat}
\phantomsection\addcontentsline{toc}{section}{\refname}\bibliography{0_home_cooperw_RStudio_WAC}

\clearpage{}

% \centerline{-- Blank Page, End of this Technical Report --}
% \vfill\eject
% \clearpage
%\addcontentsline{toc}{section}{End}


\end{document}
